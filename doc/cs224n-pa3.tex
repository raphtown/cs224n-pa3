\documentclass[12pt,a4paper]{article}
\usepackage[utf8]{inputenc}
\usepackage{amsmath}
\usepackage{amsfonts}
\usepackage{amssymb}
\usepackage{enumerate}
\usepackage{graphicx}
\usepackage{subcaption}
\usepackage{enumitem}
\usepackage{array}
\usepackage[top=1in, bottom=1.5in, left=1in, right=1in]{geometry}
\author{William Hamilton, Raphael Townshend\footnote{Both authors contributed equally to this assignment. Code and writing was contributed by both authors. Raphael focused more on the rule-based system, while William focused more on the classifier-based system.}}
\title{CS224n PA3}
\begin{document}
\maketitle
\section{Performance}
Tables 1 and 2 outline the performance of our non-baseline systems. 
\begin{table}[h!]
\centering
\begin{tabular}{c|m{40pt} m{40pt} m{40pt}| m{40pt} m{40pt} m{40pt}}
 &  & MUC &  &  & B3 &  \\ 
& Precision & Recall & F1 & Precision & Recall & F1 \\ 
\hline 
Rule-based  & • & • & • & • & • & • \\ 
Classifier-based & 0.820 & 0.695 & 0.752 & 0.776 & 0.621 & 0.690 \\ 
\hline 
\end{tabular}
\caption{Performance on the dev set.}
\end{table} 
\begin{table}[h!]
\centering
\begin{tabular}{c|m{40pt} m{40pt} m{40pt}| m{40pt} m{40pt} m{40pt}}
 &  & MUC &  &  & B3 &  \\ 
& Precision & Recall & F1 & Precision & Recall & F1 \\ 
\hline 
Rule-based  & • & • & • & • & • & • \\ 
Classifier-based & 0.823 & 0.632 & 0.715 & 0.837 & 0.610 & 0.706 \\ 
\hline 
\end{tabular}
\caption{Performance on the test set.}
\end{table} 

\section{Classifier-Based System}

This section provides details on the classifier based system.

\subsection{Description and Motivation of Features Used}

The features used in the final system are described below. 

\begin{itemize}[itemsep=-1pt]
\item
	\textbf{MDI} [IntIndicator] - The number of mentions in the document occurring between the candidate and target mentions. Motivation: captures salience of candidate. 
\item
	\textbf{HED} [IntIndicator]- Levenshtein distance between candidate and fixed mention head words (subsumes exact match).  Motivation:``fuzzy'' extension of the exact match criterion. 
\item
	\textbf{(CAND/FIXED)\_NER} [StringIndicator] - NER tag of candidate and fixed mention head words (adding the conjunction hurt the scores). Motivation: different NER types are more/less likely to act as referents.\footnote{Despite the fact that it seems strange to have an indicator feature on the target mention only, this actually helped results a marginal amount.}
\item
	\textbf{(CAND/FIXED)\_POS} [StringIndicator + Conjunction] - POS tag of candidate and fixed mention head words (and conjunction of both). Motivation: words with different POS tags are more/less likely to act as referents in general and/or co-refer.
\item
	\textbf{GEN} [Indicator] - Whether the gender of the fixed and candidate mentions match. Motivation: gender of coreferents usually correspond (except in small number of cases, e.g. ``sailboat'' referred to as ``her''). 
%\item
%	\textbf{(NUM-(MATCH/COMPAT)} [Indicator] -  Whether the candidate and fixed mentions have matching grammatical number or are compatible (e.g., grammatical number of one or both unknown). Motivation: grammatical number of co-referents almost always agree. 
\item
	\textbf{SPEAK} [Indicator] - True if both mentions are quoted,  have first-person pronoun head words, and have the same speaker. Motivation: first person pronouns almost always refer to the speaker. 
\end{itemize}
	
Some intuitively reasonable features that were examined but actually hurt results on the dev set included: a feature on the distance between the sentences in question, a feature on the compatibility of the grammatical number of the fixed and candidate mentions, and a feature on the Levenshtein distance between the entire mentions (not just head words).
Our hypothesis is that these features led to overfitting. 

\subsection{Error-Analysis}

Perhaps the most glaring systematic error in the classifier system is its tendency to chain together pronouns, regardless of their compatibility. 
For example, in the sentence ``They turned against it'', the system chains together ``They'' and ``it''.  
Or, even more egregiously, in the sentence``I thank you my pious brother.'', the system chains together all the pronouns! 
These errors could be addressed by adding a more fine-grained feature on the pronoun type. 
That is, the \textbf{(CAND/FIXED)\_POS} features could be augmented to include more more refined categories of pronouns. 
The conjunction of these more refined features would also be necessary in order to learn binary relationships (e.g.,``it'' and ``you'' are incompatible). 

The above example errors also illustrate a second systematic shortcoming of the system: it tends to chain together subjects and objects, whereas this should only happen in rare cases with reflexive verb constructions (e.g., with the ``self'' suffixed reflexive pronouns). 
This issue could be addressed by adding a conjunction feature on the grammatical role of the target and candidate mentions, and a feature indicating the presence of a reflexive verb/pronoun construction (this second feature would also be conjoined with the grammatical role conjunction).

Lastly, the system often failed to properly deal with first-person speaker pronouns despite the fact that we explicitly added a feature, \textbf{SPEAK},  to address this.
Our hypothesis is that, since this feature fires rarely in training, it has low weight (which we verified) and thus is overshadowed by other simpler features (e.g., the \textbf{HED} feature).

\subsection{Feature Ablations}

\begin{table}[h!]
\centering
\begin{tabular}{c||c|c|c|c|c|c|c|}
Features & All & -\textbf{MDI} & -\textbf{HED} &  -\textbf{NER} & -\textbf{GEN} & -\textbf{POS} & -\textbf{SPEAK}\\ 
\hline 
MUC F1& 0.752 & 0.727 & 0.424 &  0.750 & 0.744 & 0.716 & 0.751  \\ 
\hline 
B3 F1& 0.691 & 0.670 & 0.520 & 0.688 & 0.672 & 0.657 & 0.683\\ 
\hline 
\end{tabular} 
\caption{Performance on dev set with different feature (sets) ablated. For the -\textbf{POS} and -\textbf{NER} conditions we are ablating all features of that type.}
\end{table}

Ablation studies revealed that the head word edit distance, mention distance, and POS-based features were by far the most important, as the scores dropped drastically when any of these features were ablated.
This makes sense since the \textbf{MDI} feature is the only one that captures saliency, the \textbf{HED} feature captures (near)-exact matches (very important), and the POS-based features contain information about whether or not the mentions are pronouns etc. and captures some aspects of compatibility. 

That said, running the system with only these 3 most important types of features gave an MUC F1 of 0.739 and a B3 F1 of 0.669, a significant drop from the full system's performance.  Thus all the features played a part in the overall performance of the system. 




\end{document}